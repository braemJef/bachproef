%%=============================================================================
%% Voorwoord
%%=============================================================================

\chapter*{Woord vooraf}
\label{ch:voorwoord}

Voor u ligt de bachelorproef ‘JavaScript frameworks voor webapps: een vergelijkende studie’. In deze studie vergelijk ik drie verschillende JavaScript frameworks. Ik heb dit onderwerp gekozen in het kader van mijn afstudeerrichting toegepaste informatica aan de HoGent campus Aalst. Tot en met mei 2018 heb ik gewerkt en geschreven aan deze proef.

Het onderwerp voor deze bachelorproef heb ik zelf gekozen. Sinds dit jaar is mijn interesse in JavaScript frameworks zeer versterkt. We hebben in het vak webapps uitgebreid het framework Angular besproken. Na enkele weken bleek dit vak mij sterk te interesseren. Dit was ook de aanzet om dit onderwerp te kiezen.

Deze interesse was ook de reden waarom ik een stageplaats in verband met het web gekozen heb. Ik heb tijdens deze periode stage gelopen bij Codifly en heb daar gewerkt in React. Door met deze technologie te werken in de werkomgeving is mijn interesse enkel gestegen.

In de eerste plaats wil ik mijn promotor Tom Antjon bedanken voor al het geduld in deze periode. Daarnaast wil ik ook mijn co-promotor Kristof Van Miegem bedanken voor al zijn enthousiasme en inzet. Ten slotte wil ik al mijn klasgenoten bedanken die mij geholpen hebben tijdens deze proef. Het zei om vragen te beantwoorden of mijn teksten na te lezen.

%% TODO:
%% Het voorwoord is het enige deel van de bachelorproef waar je vanuit je
%% eigen standpunt (``ik-vorm'') mag schrijven. Je kan hier bv. motiveren
%% waarom jij het onderwerp wil bespreken.
%% Vergeet ook niet te bedanken wie je geholpen/gesteund/... heeft

