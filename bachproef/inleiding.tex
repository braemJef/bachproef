%%=============================================================================
%% Inleiding
%%=============================================================================

\chapter{Inleiding}
\label{ch:inleiding}

Zoals eerder in de samenvatting vermeld zijn JavaScript framewordorks sterk gegroeid de voorbije jaren. Op dit moment zijn er tientallen frameworks in development. Niet elk framework is even efficiënt in taken uitvoeren. Sommige frameworks hebben meer functionaliteit dan anderen, terwijl anderen gebruiksvriendelijker zijn voor de developer. In dit onderzoek zullen we deze verschillen proberen schetsen voor de drie populairste frameworks React, Angular en Vue. Deze zijn bepaald aan de hand van een combinatie van het aantal watchers en forks op GitHub \autocite{github_front-end_????}.

\section{Probleemstelling}
\label{sec:probleemstelling}

Met deze sterke opkomst van JavaScript frameworks is er een groot aanbod gekomen. Het voordeel van dit grote aanbod is dat er een geschikter framework gekozen kan worden voor een specifieke webapplicatie. Het grote nadeel hiervan is dat de developer (bijna) nooit de moeite doet om het meest geschikte framework te kiezen. Meestal zal deze keuze gebaseerd worden op persoonlijke voorkeur of ervaring met een bepaald framework. In deze proef proberen we te achterhalen welke voordelen en nadelen bepaalde frameworks hebben. De personen die een meerwaarde aan deze proef zullen hebben zijn:

\begin{itemize}
	\item webdevelopers
	\item de JavaScript framework developers zelf.
\end{itemize}

\section{Onderzoeksvraag}
\label{sec:onderzoeksvraag}

In dit onderzoek zullen we niet enkel performantie vergelijken maar ook pijlers zoals modulariteit, stabiliteit en nog anderen. Aan de hand van deze vergelijkingen proberen we een beter beeld te scheppen over de verschillen en gelijkenissen in deze frameworks. Welke JavaScript frameworks zijn het best resultaat gericht en development gericht?

\section{Onderzoeksdoelstelling}
\label{sec:onderzoeksdoelstelling}

Naar het einde toe van dit onderzoek proberen we een duidelijk beeld te kunnen scheppen van de verschillen en/of gelijkenissen tussen de verschillende frameworks die besproken worden. Het doel is niet om één framework als beste te beschouwen maar om de voor en nadelen van elk framework op te lijsten en te vergelijken.

\section{Opzet van deze bachelorproef}
\label{sec:opzet-bachelorproef}

% Het is gebruikelijk aan het einde van de inleiding een overzicht te
% geven van de opbouw van de rest van de tekst. Deze sectie bevat al een aanzet
% die je kan aanvullen/aanpassen in functie van je eigen tekst.

De rest van deze bachelorproef is als volgt opgebouwd:

In Hoofdstuk~\ref{ch:stand-van-zaken} wordt een overzicht gegeven van de stand van zaken binnen het onderzoeksdomein, op basis van een literatuurstudie.

In Hoofdstuk~\ref{ch:methodologie} wordt de methodologie toegelicht en worden de gebruikte onderzoekstechnieken besproken om een antwoord te kunnen formuleren op de onderzoeksvragen.

% TODO: Vul hier aan voor je eigen hoofstukken, één of twee zinnen per hoofdstuk

in Hoofdstuk~\ref{ch:onderzoek} wordt het theoretisch en praktisch onderzoek uitgevoerd en worden de resultaten weergegeven om een antwoord te kunnen formuleren op de onderzoeksvragen. 

In Hoofdstuk~\ref{ch:conclusie}, tenslotte, wordt de conclusie gegeven en een antwoord geformuleerd op de onderzoeksvragen. Daarbij wordt ook een aanzet gegeven voor toekomstig onderzoek binnen dit domein.

