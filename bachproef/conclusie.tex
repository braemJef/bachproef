%%=============================================================================
%% Conclusie
%%=============================================================================

\chapter{Conclusie}
\label{ch:conclusie}

In dit onderdeel zal de conclusie van dit onderzoek besproken worden. Er zal gesproken worden over de verschillende voor- en nadelen van elk framework.

\section{Theoretische vergelijking}

De resultaten waar we dit hoofdstuk uit afleiden zijn te vinden in hoofdstuk \ref{sec:theoretische_vergelijking}.

Uit de populariteit blijkt dat angular het hoogste scoort. We zien ook bij Google Trends \autocite{_google_2018} dat Angular stagneert of zelfs een beetje daalt op vlak van populariteit. React daarentegen is sterk in opmars en stijgt enorm. Op stackoverflow \autocite{_stackoverflow_2018} zien we dat alles enorm aan het stijgen is. De reden hiervoor is omdat React en Angular nog niet lang geleden nieuwe versies uitgebracht hebben.

Verder kunnen we de populariteit ook afleiden uit de GitHub watchers. Angular scoort op dit vlak het laagste. Dit kan erop wijzen dat Angular minder snel bugs oplost of een kleinere community heeft om problemen te verhelpen. React en Vue scoren hierop dan weer hoger en we kunnen er dus van uitgaan dat zij snellere bug fixes zien \autocite{github_front-end_????}.

Op vlak van security scoort Angular het beste, enkel dit framework beschermd de website tegen Cross Site Request Forgery en Cross Site Scripting. Bij Vue en React zal deze bescherming door de developer zelf gemaakt moeten worden.

Angular legt ook een bepaald patroon op bij de developer. Vue en React doen dit niet. Bij beginnende developers kan dit zeer voordelig zijn omdat er dan minder inconsistenties in de project indeling zitten. Bij Vue en React kan dit nogal uit de hand lopen als er geen concrete afspraken gemaakt worden rond het patroon dat men zal volgen voor de project indeling. Hierbij zijn er dus geen winnaars of verliezers. Angular voorziet een uitgebreid en makkelijk te begrijpen structuur terwijl je bij Vue en React deze structuur zelf kan bepalen.

Elk framework ondersteund een soort van unit tests en integratie tests. Het ene voorziet deze implementatie zelf terwijl je bij de andere een library moet downloaden maar dit komt op hetzelfde neer. De ondersteuning is er altijd.

\section{Praktische vergelijking}

Bij de bundle size is Vue duidelijk de winnaar. Daarna komt React en Angular zit ver achter. De reden hiervoor hebben we reeds besproken in hoofdstuk \ref{sec:grootte}. Verder in dit onderdeel zien we ook dat Vue trager is dan React en Angular. De reden hiervoor kan zijn is dat de bundle size kleiner is en dus de complexiteit ook simpelder. Dit is een speculatie en geen bewezen fijt. Wat we wel weten is dat Vue en React op een gelijkaardige manier gaan rerenderen. Dit wordt besproken in hoofdstuk \ref{sec:JavaScript_Frameworks}.

De first meaningfull paint is het snelste bij Vue. Kort daarna komt React. Ten laatste hebben we Angular dat veel achterstand heeft. De reden hiervoor wordt besproken in hoofdstuk \ref{sec:first_meaningfull_paint}. Vue is bij deze test het snelste dankzij de kleine bundel size.

Daarnaast is ook het tarief gemeten, hierbij was React de duidelijke winnaar. React was bijna dubbel zo snel dan Angular. Daarna kwam Vue die het traagste was. Vue is ten opzichte van React en Angular een recent framework. Dit kan de oorzaak zijn waarom de performantie nog niet compleet op punt is. React en Angular zijn ook veel stabieler, dit kunnen we zien aan de kleine standaarddeviatie in tabellen \ref{table:resultaten_100_elementen}, \ref{table:resultaten_500_elementen} en \ref{table:resultaten_1000_elementen}.

\section{Samenvatting conclusie}

Voor developers die een gestructureerd framework willen dat voor hen al grotendeels errors kan voorkomen. Een framework dat performant is en al voorzieningen heeft voor security is Angular de geschikte keuze.

Als de voorkeur eerder gaat naar flexibiliteit en top performantie waarbij er zelf veel beslist en gekozen kan worden is React de geschikte keuze. De developer zal echter meer werk moeten verichten.

Bij Vue is de performantie dan wat lager maar de bundle size is het kleinst. De grote community erachter en populariteit zullen ofwel sterk toenemen ofwel dalen. Als deze blijft groeien zoals ze nu is zullen we enkel verbeteringen zien. De performantie zal zeker nog aan gewerkt worden dan.

Uiteindelijk is het een balans tussen ervaring en het geschikte framework kiezen. De competitie bij deze frameworks is groot en de verschillen ertussen zullen als maar kleiner worden.

%% TODO: Trek een duidelijke conclusie, in de vorm van een antwoord op de
%% onderzoeksvra(a)g(en). Wat was jouw bijdrage aan het onderzoeksdomein en
%% hoe biedt dit meerwaarde aan het vakgebied/doelgroep? Reflecteer kritisch
%% over het resultaat. Had je deze uitkomst verwacht? Zijn er zaken die nog
%% niet duidelijk zijn? Heeft het onderzoek geleid tot nieuwe vragen die
%% uitnodigen tot verder onderzoek?