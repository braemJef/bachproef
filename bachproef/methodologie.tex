%%=============================================================================
%% Methodologie
%%=============================================================================

\chapter{Methodologie}
\label{ch:methodologie}

In dit onderdeel zal de methodologie die in dit onderzoek gebruikt is uitgebreid besproken worden. 

Eerst en vooral is de kennis bijgeschaafd over dit onderwerp. Zo veel mogelijk artikels in verband met JavaScript, React, Angular en Vue gelezen. Ook wat informatie over Node.js opgezocht.

Hierna is de standvanzaken of literatuurstudie gemaakt. Hierin is alles zo chronologisch mogelijk uitgelegd. Er is begonnen met JavaScript zelf uit te leggen om een algemeen beeld te scheppen over de technologie. Hierna is er dieper ingegaan op wat een framework eigenlijk is en waarom het belangrijk is.

In de stand van zaken hebben we ook onderzocht hoe we een framework het beste kunnen vergelijken met elkaar. Er zijn bepaalde criterea die van belang zijn. Deze hebben we dan opgedeeld in twee delen. Theoretische en praktische vergelijking. Door eerst de theoretische vergelijking uit te voeren krijgen we een beter beeld van de frameworks zelf. Hierna werd de praktische vergelijking gedaan. Hier waren af en toe wat abnormaliteiten. Door het theoretische deel eerst te doen waren deze logisch te verklaren.

Na de stand van zaken te onderzoeken zijn de testapplicaties geschreven. Tijdens het schrijven werd er rekening gehouden om het zo gelijk mogelijk te houden over de 3 frameworks. Elke applicatie heeft dezelfde functionaliteiten. Er is ook rekening gehouden met de literatuurstudie waar we lifecycle methoden besproken hebben. Dit onderdeel was cruciaal om de benchmarks later correct te laten verlopen.

Hierna is het onderzoek van start gegaan. Door hier ook weer eerst het theoretische deel te onderzoeken kon het praktische deel beter verstaan worden. De resultaten zijn hierna grondig bewaard en in de bijlagen bijgevoegd.

Ten slotte is er een conclusie opgesteld aan de hand van de literatuurstudie, de theoretische vergelijking en praktische testen.

%% TODO: Hoe ben je te werk gegaan? Verdeel je onderzoek in grote fasen, en
%% licht in elke fase toe welke stappen je gevolgd hebt. Verantwoord waarom je
%% op deze manier te werk gegaan bent. Je moet kunnen aantonen dat je de best
%% mogelijke manier toegepast hebt om een antwoord te vinden op de
%% onderzoeksvraag.


