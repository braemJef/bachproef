%==============================================================================
% Sjabloon onderzoeksvoorstel bachelorproef
%==============================================================================
% Gebaseerd op LaTeX-sjabloon ‘Stylish Article’ (zie voorstel.cls)
% Auteur: Jens Buysse, Bert Van Vreckem

% TODO: Compileren document:
% 1) Vervang ‘naam_voornaam’ in de bestandsnaam door je eigen naam, bv.
%    buysse_jens_voorstel.tex
% 2) latexmk -pdf naam_voornaam_voorstel.tex
% 3) biber naam_voornaam_voorstel
% 4) latexmk -pdf naam_voornaam_voorstel.tex (1 keer)

\documentclass[fleqn,10pt]{voorstel}

%------------------------------------------------------------------------------
% Metadata over het artikel
%------------------------------------------------------------------------------

\JournalInfo{HoGent Bedrijf en Organisatie} % Journal information
\Archive{Onderzoekstechnieken 2016 - 2017} % Additional notes (e.g. copyright, DOI, review/research article)

%---------- Titel & auteur ----------------------------------------------------

% TODO: geef werktitel van je eigen voorstel op
\PaperTitle{JavaScript frameworks voor webapps: een vergelijkende studie}
\PaperType{Onderzoeksvoorstel Bachelorproef} % Type document

% TODO: vul je eigen naam in als auteur, geef ook je emailadres mee!
\Authors{Jef Braem\textsuperscript{1}} % Authors
\affiliation{\textbf{Contact:}
  \textsuperscript{1} \href{mailto:jef.braem.w9675@student.hogent.be}{jef.braem.w9675@student.hogent.be}}

%---------- Abstract ----------------------------------------------------------

  \Abstract{Welk JavaScript framework gebruik ik voor mijn project? Hoe is de performance of schaalbaarheid? In deze scriptie ga ik een vergelijkende studie maken tussen verschillende JavaScript frameworks. Ik zal ze testen op performantie (snelheid, geheugengebruik). Maar dit alleen zou niet genoeg zijn om een goede keuze op te baseren. Ik zal dus ook bij elk onderzocht framework een kleine applicatie maken en hierbij het MVC patroon implementeren. Hiermee krijg ik een beeld over hoe elk framework werkt, en geef ik hier ook mijn mening over. Het onderzoek kan dienen als basis voor de keuze welk JavaScript framework te gebruiken bij een project. Ik verwacht dat de performantie van de verschillende frameworks niet ver van elkaar zal liggen omdat er anders geen competitie kan zijn.}

%---------- Onderzoeksdomein en sleutelwoorden --------------------------------
% TODO: Sleutelwoorden:
%
% Het eerste sleutelwoord beschrijft het onderzoeksdomein. Je kan kiezen uit
% deze lijst:
%
% - Mobiele applicatieontwikkeling
% - Webapplicatieontwikkeling
% - Applicatieontwikkeling (andere)
% - Systeem- en netwerkbeheer
% - Mainframe
% - E-business
% - Databanken en big data
% - Machine learning en kunstmatige intelligentie
% - Andere (specifieer)
%
% De andere sleutelwoorden zijn vrij te kiezen

\Keywords{Onderzoeksdomein. JavaScript --- Frameworks --- Webapplicaties} % Keywords
\newcommand{\keywordname}{Sleutelwoorden} % Defines the keywords heading name

%---------- Titel, inhoud -----------------------------------------------------
\begin{document}

\flushbottom % Makes all text pages the same height
\maketitle % Print the title and abstract box
\thispagestyle{empty} % Removes page numbering from the first page

%------------------------------------------------------------------------------
% Hoofdtekst
%------------------------------------------------------------------------------

%---------- Inleiding ---------------------------------------------------------

\section{Introductie} % The \section*{} command stops section numbering
\label{sec:introductie}
Er zijn reeds vele JavaScript frameworks in omloop. Als je aan een project begint weet je soms niet goed welke de beste keuze is en neem je waarschijnlijk het framework waar je het meest mee vertrouwt bent. Ik vroeg mij af, zou het niet efficiënter zijn om het meteen het meest geschikte framework te kiezen? Natuurlijk is het efficiënter om met iets te werken dat je kent. Maar als het een groot project is misschien niet.\\
\\
Ik doe dit onderzoek omdat het vak webapps mij interesseert. Hierbij gebruiken we Angular 4. Ook heb ik al een kleine chat applicatie gemaakt als test voor mijn stageplaats gemaakt in react. Door deze twee gebeurtenissen is mijn interesse in web development gepiekt. Toen ik meer zoekwerk deed vond ik veel meer frameworks. Ik vroeg mij dus af welke beter is.\\
\\
Met dit onderzoek wil ik een beter beeld vormen van 3 frameworks. Hiervoor ga ik mij baseren op de populariteit van JavaScript frameworks en heb besloten de 3 meest gebruikte te bestuderen. Het beste voor dit soort onderzoek is zo veel mogelijk verschillende frameworks, maar ik denk dat de tijd mij niet zal toestaan om meer dan 3 te bestuderen.\\
\\
Bij dit onderzoek stel ik mij de volgende vragen:
\begin{itemize}
	\item welk framework reageert het snelst?
	\item welk framework gebruikt het minste geheugen?
	\item waar implementeer je MVC het makkelijkst naar mijn gevoel?
\end{itemize}

%---------- Stand van zaken ---------------------------------------------------

\section{Literatuurstudie}
\label{sec:literatuurstudie}

\subsection{Wat zijn JavaScript frameworks?}

Een framework is een soort van skelet voor je project. Het heeft al een bepaalde lijst van functionaliteiten ter beschikking. Dankzij een framework zal een project opbouwen veel sneller verlopen.\\
\\
Bij web development is er ook sprake van het multiplatformprobleem. JavaScript is zeer browseronafhankelijk maar het wordt veel gebruik tom de DOM te manipuleren. JavaScript frameworks lossen dit op door extra code voor verschillende browsers te genereren. Zo hoeft de ontwikkelaar zich hier geen zorgen meer over te maken. \autocite{javascriptframework}

\subsection{Gelijkaardige onderzoeken}

Er zijn zeker al gelijkaardige onderzoeken gedaan over dit onderwerp. Hetgeen me  opviel is dat de meeste onderzoeen die al uitgevoerd zijn vooral over performantie van JavaScript frameworks gaan. Natuurlijk zal ik ook een deel van mijn tijd besteden aan performantie maar ik zou graag niet compleet in die richting gaan en ook een andere kijk op JavaScript frameworks hebben. Wat de moeilijkheidsgraad was om MVC te implementeren, hoe de leercurve was persoonlijk voor mij. Dit kan dan een beter beeld scheppen voor mensen die deze scriptie lezen.

% Voor literatuurverwijzingen zijn er twee belangrijke commando's:
% \autocite{KEY} => (Auteur, jaartal) Gebruik dit als de naam van de auteur
%   geen onderdeel is van de zin.
% \textcite{KEY} => Auteur (jaartal)  Gebruik dit als de auteursnaam wel een
%   functie heeft in de zin (bv. ``Uit onderzoek door Doll & Hill (1954) bleek
%   ...'')

%---------- Methodologie ------------------------------------------------------
\section{Methodologie}
\label{sec:methodologie}

In dit onderzoek zal ik starten met een uitgebreide literatuurstudie over JavaScript frameworks en over de gekozen frameworks. Daarna zal ik voor alle technologiën de performantie onderzoeken. Ten slotte zal ik voor elke framework één functioneel identieke MVC applicatie schrijven.

%---------- Verwachte resultaten ----------------------------------------------
\section{Verwachte resultaten}
\label{sec:verwachte_resultaten}

Ik verwacht dat de performantie van de frameworks niet ver van elkaar zal liggen. Computers zijn tegenwoordig zeer snel en er zullen dus maar kleine verschillen zijn.

%---------- Verwachte conclusies ----------------------------------------------
\section{Verwachte conclusies}
\label{sec:verwachte_conclusies}

Ik verwacht dat er wel een duidelijke conclusie zal zijn. Hiermee bedoel ik dat ik hoop op één framework waar de implementatie en/of leercurve veel makkeljiker of sneller zullen gaan dan bij de anderen. Aangezien ik al gewerkt heb met angular en met react zal ik hier mee rekening houden.

%------------------------------------------------------------------------------
% Referentielijst
%------------------------------------------------------------------------------
% TODO: de gerefereerde werken moeten in BibTeX-bestand ``biblio.bib''
% voorkomen. Gebruik JabRef om je bibliografie bij te houden en vergeet niet
% om compatibiliteit met Biber/BibLaTeX aan te zetten (File > Switch to
% BibLaTeX mode)

\phantomsection
\printbibliography[heading=bibintoc]

\end{document}
